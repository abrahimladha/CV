\documentclass[11pt]{article} %set margins to .5 instead of 1
\usepackage[margin=1.0in]{geometry}
%\usepackage{enumitem}
\usepackage{hyperref}
\usepackage{enumerate}
\parindent=0pt
\parskip=\medskipamount
\begin{document}
{\huge Abrahim K. Ladha}\\
\hrule
%contact info
\section*{CONTACT INFO}
\begin{itemize}
  \setlength{\itemsep}{1pt}
  \setlength{\parskip}{0pt}
  \setlength{\parsep}{0pt}
\item \emph{Phone:} 912-856-8626
\item \emph{Address:} 418 East 56th Street
\item \emph{Email:}  \href{mailto:abrahimladha@protonmail.ch}{abrahimladha@protonmail.ch} or \href{mailto:al8292@stu.armstrong.edu}{al8292@stu.armstrong.edu}%make those links
\item \emph{Website:}  \url{https://ladha.me} %also make link
\end{itemize}
\section*{EDUCATION}
\hangindent=2em
\hangafter=1
\emph{Georgia Institute of Technology} Fall 2016 - Present\\
Mathematics Major with minors in Computer Science and Physics planned GPA 

\hangindent=2em
\hangafter=1
\emph{Armstrong State University} Fall 2016 - Spring 2016\\
Transferred to Georgia Tech after two years GPA 3.78/4.0

\hangindent=2em
\hangafter=1
\emph{H. V. Jenkins High School, Engineering Academy} 2010 - 2014\\
Graduated AP Scholar with Distinction
\section*{EXPERIENCE}
\hangindent=2em
\hangafter=1
\emph{Armstrong Research Secure Multiparty Computation Summer 2016}\\
bluurrrr\

\hangindent=2em
\hangafter=1
\emph{Programming Competition Coordinator Spring 2016}\\
I organize and run practice sessions twice a week for Armstrong students to prepare for competitions such as the ACM International Collegiate Programming Contest and other small regional competitions.\\

\hangindent=2em
\hangafter=1
\emph{Armstrong Research Topological Graph Theory Spring 2016}\\
Worked under Dr. Lambert on several open problems in topological graph theory. We spent several weeks developing bounds for the crossing number of the Paley graph on 13 vertices. \\

\hangindent=2em
\hangafter=1
\emph{Student Worker Summer 2015 - Spring 2016}\\
I oversaw and helped maintain equipment and computers in Armstrong's Computer Science labs, including 3D printers, robotics, and some haptics equipment. I was also responsible for helping to organize and plan future events in the Computer Science Department. \\

\hangindent=2em
\hangafter=1
\emph{Armstrong Research The N-Queens Summer 2014 - Present}\\
Worked under Dr. Brown on the N-Queens Problem as a combinatorics researcher as part of the STEP Summer program. The research continued for the 2014-2015 year allowing me to write a paper, and attend several conferences and events to present my work.

\hangindent=2em
\hangafter=1
\emph{Georgia Tech CEISMIC Instructor Spring 2013 - Summer 2015}\\
Worked as an Instructor for the CEISMIC program at the Georgia Tech Savannah Campus. My duties were to teach basic paradigms in programming to K-5 Students during the Summer, and Girl Scout Troops during the school year. The tools used for instruction were Scratch and Construct 2.
\section*{PRESENTATIONS}
\begin{itemize}
  \setlength{\itemsep}{1pt}
  \setlength{\parskip}{0pt}
  \setlength{\parsep}{0pt}

\item \textit{State Space Graphs and the $N$-Queens Problem} 
  \begin{itemize}
	  	\setlength{\itemsep}{1pt}
  		\setlength{\parskip}{0pt}
 	 	\setlength{\parsep}{0pt}
	\item 95th MAA Southeastern Section Meeting (future, abstract submitted)
	\item Supported by the Armstrong Math Club
	\item March 26th 2015
	\end{itemize}
	  \item \textit{How to make a Video Game! ACM Workshop for Students} 
  \begin{itemize}
	  	\setlength{\itemsep}{1pt}
  		\setlength{\parskip}{0pt}
 	 	\setlength{\parsep}{0pt}
		\item Conducted Student ACM Lecture Series Number 4
	\item Supported by the ACM Student Chapter
	\item February 26th 2016
	\end{itemize}
\item \textit{State Space Graphs and the $N$-Queens Problem} 

\begin{itemize}
	  	\setlength{\itemsep}{1pt}
  		\setlength{\parskip}{0pt}
 	 	\setlength{\parsep}{0pt}
	\item Eagle Undergraduate Mathematics Conference (Georgia Southern University)
	\item Supported by the Armstrong Math Club
	\item February 13th 2015
	\end{itemize}
\item \textit{DuckDuckHack - So you want to write a search engine? ACM Workshop for Students}
\begin{itemize}
	  	\setlength{\itemsep}{1pt}
  		\setlength{\parskip}{0pt}
 	 	\setlength{\parsep}{0pt}
	\item Conducted Student ACM Lecture Series Number 1
	\item Supported by DuckDuckGo and ACM Student Chapter
	\item September 22nd 2015
	\end{itemize}
\item \textit{A Comparision of Game Engines (Poster)}
\begin{itemize}
	  	\setlength{\itemsep}{1pt}
  		\setlength{\parskip}{0pt}
 	 	\setlength{\parsep}{0pt}
	\item TechFest 2015
	\item Supported by the Computer Science department at Armstrong
	\item April 17th 2015
	\end{itemize}
\item \textit{Exploring A Derivation of the The $N$-Queens Problem.} 
\begin{itemize}
	  	\setlength{\itemsep}{1pt}
  		\setlength{\parskip}{0pt}
 	 	\setlength{\parsep}{0pt}
	\item Armstrong Student Scholarship Symposium
	\item Sponsored by Armstrong Undergraduate Research
	\item April 17th 2015
	\end{itemize}
\item \textit{Calculating and Optimization of Every Possibility of a Derivation of the $N$-Queens Problem.}
\begin{itemize}
	  	\setlength{\itemsep}{1pt}
  		\setlength{\parskip}{0pt}
 	 	\setlength{\parsep}{0pt}
	\item Kennesaw Undergraduate Research Conference
	\item Sponsored by Armstrong Undergraduate Research
	\item October 11th 2014
	\end{itemize}

\item \textit{The $N$-Queens Problem.}
	\begin{itemize}
	  	\setlength{\itemsep}{1pt}
  		\setlength{\parskip}{0pt}
 	 	\setlength{\parsep}{0pt}
	\item End of STEP Program
	\item Sponsored by NSF-STEP 0856593
	\item July 25th 2014
	\end{itemize}
\end{itemize}

\section*{PREPRINTS}
\hangindent=2em
\hangafter=1
\emph{Hypothetical Problems concerning the Theory of
Relativity on Cryptographic Currency
Implementations} - 
Bitcoin has demonstrated there are many security
improvements applicable to normal currency. As the human race
expands and we colonize other planets, we have to consider how
we are going to extend integral parts of of society, and that
includes our currency system. Information transferring does not
scale well with very large distances, entirely due to physical
limitations. For example, there is a maximum speed that any
information can travel, and it cannot be faster than the speed
of light. In this paper we take these physical limitations into
account to give treatment to the following question. Can a single
crypto-currency be used across the entire universe? Trivially
many currencies can be used with exchange rates but we will
try to avoid this as our solution. The idea of this paper was
inspired by a paper titled ”The Theory of Interstellar Trade” by
Paul Krugman, a Nobel Economist. 
\url{http://arxiv.org/abs/1604.04265}

\hangindent=2em
\hangafter=1
\emph{Exploring mod-2 $n$-queens games (submitted for publication)}\\
We introduce a two player game on an $n \times n$ chessboard where queens are placed by alternating turns on a chessboard square whose availability is determined by the number of queens already on the board which can attack that square modulo two. The game is explored along with some variations and its complexity.\\
\url{http://arxiv.org/abs/1510.02875}

\hangindent=2em
\hangafter=1
Convergence Zine\\
\emph{"September 2014: GLITCH"} ​(pages 7­8, 15-­16, 31-­32). \url{http://issuu.com/convergencezine/docs/september2014}
\\
\section*{AWARDS}
\begin{itemize}
  \setlength{\itemsep}{1pt}
  \setlength{\parskip}{0pt}
  \setlength{\parsep}{0pt}
\item GA Power Research Scholar 2016
\item Jesse Shearhouse Scholarship Spring 2016
\item Eagle Undergraduate Mathematics Conference 3rd Place, 2016
\item ACM ICPC 4th Place, College of Charleston location, Division II, 2015
\item Dean's List Summer 2015
\item Dean's List Spring 2015
\item Dean's List Fall 2014
\item Video Game Design 3rd Place, State level, 2014
\item Video Game Design 6th Place, State level, 2013
\item Video Game Design 9th Place, State level, 2012
\end{itemize}
\section*{HOBBIES}
\begin{itemize}
  \setlength{\itemsep}{1pt}
  \setlength{\parskip}{0pt}
  \setlength{\parsep}{0pt}
\item Decentralized cryptographic currency systems and their computing needs
\item Designing Algorithms to corrupt data structures of image and 3D file formats for artistic purposes \url{http://ladha.me/art}
\item I enjoy solving puzzle cubes of varying sizes and shapes
\item Mathematical topics I enjoy include Graph Theory, Mathematical Physics, Formal Logic, Abstract Computer Science, and Discrete Models of Quantum Mechanics.
\end{itemize}
\end{document}
