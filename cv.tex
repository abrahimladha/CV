\documentclass[11pt]{article} %set margins to .5 instead of 1
\usepackage[margin=1.0in]{geometry}
%\usepackage{enumitem}
\usepackage{hyperref}
\parindent=0pt
\parskip=\medskipamount
\begin{document}
{\huge Abrahim K. Ladha}\\
\hrule
%contact info
\section*{CONTACT INFO}
\begin{itemize}
  \setlength{\itemsep}{1pt}
  \setlength{\parskip}{0pt}
  \setlength{\parsep}{0pt}
\item \emph{Phone:} 912-856-8626
\item \emph{Address:} 418 East 56th Street
\item \emph{Email:}  \href{mailto:abrahimladha@protonmail.ch}{abrahimladha@protonmail.ch} or \href{mailto:al8292@stu.armstrong.edu}{al8292@stu.armstrong.edu}%make those links
\item \emph{Website:}  \url{https://vgd.me/abrahimladha} %also make link
\end{itemize}
\section*{EDUCATION}
\hangindent=2em
\hangafter=1
\emph{Armstrong State University} 2014 - Present\\
Dual Mathematics/Computer Science Major and Physics Minor. GPA 3.71/4.0

\hangindent=2em
\hangafter=1
\emph{H. V. Jenkins High School, Engineering Academy} 2010 - 2014\\
Graduated AP Scholar with Distinction
\section*{EXPERIENCE}
\hangindent=2em
\hangafter=1
\emph{Student Worker 2015 - Present}\\
I currently oversee and maintain equipment and computers in our Computer Science labs, including 3D printers, robotics, and some haptics equipment. \\

\hangindent=2em
\hangafter=1
\emph{Armstrong Research 2014 - 2015}\\
Worked under Dr. Brown on the N-Queens Problem as a combinatorics researcher as part of the STEP Summer program. The research continued for the 2014-2015 year allowing me to write a paper, and attend the Kennesaw Undergraduate Research Conference and present my work. \\

\hangindent=2em
\hangafter=1
\emph{Georgia Tech CEISMIC Instructor 2013 - 2015}\\
Worked as an Instructor for the CEISMIC program. My duties were to teach basic paradigms in programming to K-5 Students during the Summer, and Girl Scout Troops during the school year.
\section*{RESEARCH SKILLS}
\begin{itemize}
  \setlength{\itemsep}{1pt}
  \setlength{\parskip}{0pt}
  \setlength{\parsep}{0pt}
\item Data visualization and recursive algoritm design in Mathematica
\item Extensive knowledge of Unix/Linux based operating systems.
\item Extensive knowledge of 3D professional hardware based rendering pipelines
\item Programming skills in C, Java, Mathematica, HTML, \LaTeX , and bash scripting
\end{itemize}
\section*{PREPRINTS}
\hangindent=2em
\hangafter=1
Proofs Textbook\\
I wanted to try to improve my mathematical writing, and  {\LaTeX} skills, so I tried writing my own proofs textbook. It is mostly satirical. Unfinished draft here: \url{https://vgd.me/abrahimladha/files/proofs.pdf} 

\hangindent=2em
\hangafter=1
Exploring mod-2 $n$-queens games (submitted for publication)\\
We introduce a two player game on an $n \times n$ chessboard where queens are placed by alternating turns on a chessboard square whose availability is determined by the number of queens already on the board which can attack that square modulo two. The game is explored along with some variations and its complexity.
\url{http://arxiv.org/abs/1510.02875}

\hangindent=2em
\hangafter=1
Convergence Zine\\
\emph{"September 2014: GLITCH"} ​(pages 7­8, 15­16, 31­32). \url{http://issuu.com/convergencezine/docs/september2014}


\section*{HOBBIES}
\begin{itemize}
  \setlength{\itemsep}{1pt}
  \setlength{\parskip}{0pt}
  \setlength{\parsep}{0pt}
\item Experience with Decentralized cryptographic currency systems and their computing needs
\item Designing Algorithms to corrupt data structures of image and 3D file formats for artistic purposes
\item I enjoy solving puzzle cubes of varying sizes and shapes.
\item Mathematical topics I enjoy include Graph Theory, Mathematical Physics, Formal Logic, and Abstract Computer Science.
\end{itemize}
\end{document}